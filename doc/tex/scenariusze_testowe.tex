\chapter[Scenariusze testowe][Scenariusze testowe]{Scenariusze testowe}
Testowanie oprogramowania będzie polegało na~wywoływaniu szeregu scenariuszy
powiązanych z~możliwymi usterkami występującymi podczas działania systemu
zarówno z~powodu wystąpienia sytuacji nadzwyczajnych, jak~i~z~powodów
zewnętrznych, nie~związanych z błędami w~samym oprogramowaniu. Scenariusze będą
polegały na~wywoływaniu operacji po~stronie klienta, a~następnie symulacji
objawów usterek związanych z~zadanym scenariuszem testowym; weryfikacja testu
odbywać się~będzie na~podstawie porównania reakcji systemu z~oczekiwanymi
poprawnymi reakcjami na~wysłanie żądania. Potencjalne błędy, które testy będą
musiały sprawdzić, obejmują w~szczególności problemy synchronizacyjne:
opóźnienia oraz desynchronizację, ale~również najczęściej spotykane usterki
nie~powiązane z~zagadnieniami rozproszonymi.

\begin{table}[h]
\begin{center}
	\begin{tabular}{|p{4cm} | p{4cm} | p{6cm} |}
		\hline
		\textbf{Scenariusz} & \textbf{Możliwe powody} &	\textbf{Rozwiązanie}	\\	\hline
		Timeout             & Podany serwer jest niedostępny &	Ponowne przeszukanie
		puli dostępnych serwerów \\ \hline
		Odpowiedź: IP serwerów typu Master i Shadow & Klient podłączył się~do~innego
		serwera niż~master & Nawiązanie połączenia z~masterem na~podstawie otrzymanych
		danych \\ \hline
		Klient otrzymuje niewłaściwe adresy - próbuje połączyć się~do~serwera
		o~niepoprawnym~typie & Opóźnienie propagacji, zmiana serwera master.
		& Ponowne odpytanie serwerów ze~znanej puli shadow. \\ \hline
	\end{tabular}
\end{center}
\caption{Scenariusze dla operacji \emph{connect (Client -> Server)}}
\end{table}

\begin{table}[h]
\begin{center}
	\begin{tabular}{|p{4cm} | p{4cm} | p{6cm} |}
		\hline
		\textbf{Scenariusz} & \textbf{Możliwe powody} &	\textbf{Rozwiązanie}	\\	\hline
		File not found      & Master nie może znaleźć pliku w~bazie lub plik ma
		ustawiony tryb do~usunięcia &	(Klient) Sprawdzić poprawność ścieżki oraz
		sprawdzić listing plików \\ \hline
		Slave down (timeout) & Podany serwer jest niedostępny & Wykonanie zwrotnego
		zapytania do mastera w celu wybrania nowego slave’a \\ \hline
		Not enough space (u klienta) & Brak miejsca u~klienta & Zwrócenie błędu \\
		\hline
		
	\end{tabular}
\end{center}
\caption{Scenariusze dla operacji \emph{get}}
\end{table}

\begin{table}[h]
\begin{center}
	\begin{tabular}{|p{4cm} | p{4cm} | p{6cm} |}
		\hline
		\textbf{Scenariusz} & \textbf{Możliwe powody} &	\textbf{Rozwiązanie}	\\	\hline
		Not enough space & Zapełniona przestrzeń dyskowa slave’ów & Usunąć pliki
		z~systemu \\ \hline
		(M-S) Timeout & Wybrany serwer jest niedostępny & Master wybiera nowego
		slave’a do~puli replikacyjnej i~deaktywuje niedostępnego \\ \hline
		(C-S) Timeout & Wybrany slave jest niedostępny & Klient ponawia próbę wysłania
		pliku za~pośrednictwem mastera. Dodatkowo: master wybiera dodatkowy slave
		do~replikacji i~informuje całą resztę o~zmianie lokalizacji repliki głównej.
		\\ \hline
		(S-S) Timeout & Połączenie replikacyjne nie~może zostać ustabilizowane. &
		Master zostaje poinformowany o~zaburzeniu procesu replikacyjnego i~wybiera
		nowe brakujące slave’y do~replikacji. \\ \hline
	\end{tabular}
\end{center}
\caption{Scenariusze dla operacji \emph{put}}
\end{table}

\makeatletter
\setlength{\@fptop}{0pt}
\makeatother
\begin{table}[h]
	\begin{tabular}{|p{4cm} | p{4cm} | p{6cm} |}
		\hline
		\textbf{Scenariusz} & \textbf{Możliwe powody} &	\textbf{Rozwiązanie}	\\	\hline
		File not found. & Plik nie istnieje w systemie. & Zwrócenie błędu \\ \hline
		(M-SM) Timeout & Serwer poboczny jest niedostępny. & Master wywołuje proces
		elekcji serwera pobocznego. \\ \hline
		(M-S) Timeout & Serwer magazynujący jest niedostępny. & Master zamyka
		połączenie; plik zostanie usunięty z~serwera magazynującego podczas kolejnego
		połączenia. \\ \hline
		Master ulega awarii w~trakcie oczekiwania na~slave’y & & Serwer poboczny
		wykonuje zapytanie o pliki do usunięcia z bazy i wznawia proces. \\ \hline
	\end{tabular}
\caption{Scenariusze dla operacji \emph{rm}}
\end{table}