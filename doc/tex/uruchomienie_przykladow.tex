\chapter[Uruchomienie dostępnych w sieci przykładów demonstrujących
wykorzystanie Apache Thrift][Uruchomienie dostępnych w sieci przykładów
demonstrujących wykorzystanie Apache Thrift]{Uruchomienie dostępnych w sieci
przykładów demonstrujących wykorzystanie Apache Thrift}
Jednym z~etapów dobrego poznania Apache Thrift, było uruchomienie dostępnych
w~sieci przykładów wykorzystujących to~narzędzie. Kolejnym krokiem, była
implementacja prostej usługi przypominającej bardzo uproszczone zadanie
projektowe. Uruchomionym przykładem demonstrującym wykorzystanie Apache Thrift
był ,,Thrift by Example'' ze strony:
\emph{http://thrift-tutorial.readthedocs.org/en/latest/usage-example.html}.

\vspace{5mm}
Opisano sposób realizacji prostej usługi zapewniającej operacje matematyczne.
Przykład wprowadza do:
\begin{itemize}
  \item składni oraz możliwości języka opisu interfejsu dla narzędzia - Thrift
  IDL,
  \item sposobu użycia kompilatora Thrift IDL,
  \item sposobu implementacji aplikacji klienta w języku Java,
  \item sposobu implementacji aplikacji serwera wraz z odpowiednimi klasami
  realizującymi usługę w języku Java,
  \item przykład dodatkowo opisuje sposób nawiązywania bezpiecznych połaczeń
  (przy pomocy SSL).
\end{itemize}

\vspace{5mm}
Przykładowa usługa umożliwia klientowi zapytanie serwera o wynik operacji -
w~ramach zdalnego wywołania procedury klient przekazywał argumenty oraz typ
operacji. Dodatkowo wykorzystano mechanizmy Apache Thrift umożliwiające
rzucanie wyjątków (na~przykład w przypadku próby dzielenia przez 0).

\vspace{5mm}
Demonstracja Thrift by Example wprowadza w Apache Thrift, pokazuje przykładowe
zastosowanie i~proponuje przykładową implementację wykorzystującą to~narzędzie.

\vspace{5mm}
Drugim elementem wprowadzenia do Apache Thrifta była samodzielna implementacja
bardzo uproszczonej wersji usługi, o której mowa w zadaniu projektowym
-~rozproszonego systemu plików. W~ramach tych prac powstały dwie grupy usług:
NamingService oraz StorageService, które odpowiadają kolejno warstwie usług
lokalizacyjno-nazewniczej oraz warstwie właściwego przechowywania.

\hspace*{-\parindent}%
\begin{minipage}{\linewidth}
\begin{lstlisting}[
language=C,
style=incode,
caption={Interfejs serwera nazewniczego},
morekeywords={typedef,int,service,string,namespace}]
namespace java NamingService
typedef i32 int

service NamingService
{
	int put(1:string fileName),
	int get(1:string fileName)
}
\end{lstlisting}
\end{minipage}

\hspace*{-\parindent}%
\begin{minipage}{\linewidth}
\begin{lstlisting}[
language=C,
style=incode,
caption={Interfejs serwera przechowującego dane},
morekeywords={typedef,int,service,string,namespace,byte}]
namespace java StorageService
typedef i32 int

service StorageService
{
        void putFile(1:int fileId, 2:list<byte> body),
        list<byte> getFile(1:int fileId)
}
\end{lstlisting}
\end{minipage}

\vspace{5mm}
Aplikacja kliencka, gdy~chce wysłać plik do~systemu plików, najpierw łączy
się~z~serwerem nazewniczym -~w~tym~momencie zostaje zrealizowane zarejestrowanie
nazwy oraz przypisanie identyfikatora pliku. W~drugim kroku aplikacja kliencka
łączy się~z~serwerem przechowującym dane, przesyła identyfikator pliku (nadany
przez serwer nazewniczy) oraz kolekcje bajtów (ciało pliku).
