\chapter[Organizacja środowiska programistycznego projektu][Organizacja
środowiska programistycznego projektu]{Organizacja środowiska programistycznego
projektu}
Podczas wykonywania projektu zostały wykorzystane następujące narzędzia:
\begin{itemize}
  \item \textbf{Git} - jako system kontroli wersji,
  \item \textbf{GitHub} - jako zdalne repozytorium dla narzędzia git,
  \item \textbf{Google Docs} - do~wspólnego tworzenia wstępnej dokumentacji
  projektowej,
  \item \textbf{Latex} - do~przygotowania dokumentacji projektowej,
  \item \textbf{Redmine} - do zarządzania projektem.
\end{itemize}

\vspace{5mm}
Wybór gita jako systemu kontroli wersji był podyktowany prostotą jego
użytkowania oraz wysoką funkcjonalnością. Dodatkowo, wszyscy członkowie zespołu
znają to~narzędzie.

\vspace{5mm}
Kolejny wybór dotyczył zdalnego repozytorium, na~którym miały by~być
przechowywane zasoby z~gita. Wybór padł na~serwis GitHub, gdyż oferuje
on~darmowe usługi dla~projektów niekomercyjnych oraz udostępnia wiele danych
statystycznych pomocnych przy zarządzaniu projektem (np.~wykres aktywności).

\vspace{5mm}
Jako narzędzie do~zarządzania projektem wybrano system Redmine. Jego głównymi
zaletami są:
\begin{itemize}
  \item wysoka funkcjonalność,
  \item możliwość integracji z~innymi systemami (np.~z~systemem kontroli wersji
  git),
  \item łatwość użytkowania.
\end{itemize}
