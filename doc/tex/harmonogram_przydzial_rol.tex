\chapter[Harmonogram realizacji projektu. Przydział ról w projekcie][Harmonogram
realizacji projektu. Przydział ról w projekcie]{Harmonogram realizacji projektu.
Przydział ról w projekcie}
Podczas pierwszego spotkania projektowego ustalono podział ról oraz terminy
przyszłych spotkań. Podział obowiązków ulegał modyfikacjom. Ostatecznie
ustalono następujący przydział ról:
\begin{itemize}
  \item kierownik projektu: Jacek Witkowski,
  \item architekt: Marcin Swend,
  \item zarządca repozytorium: Jacek Witkowski,
  \item dokumentalista: Jacek Witkowski,
  \item tester: Marcin Dzieżyc,
  \item handlowiec: Mateusz Statkiewicz.
\end{itemize}

Oprócz powyżej przydzielonych ról specjalnych, każdy z członków zespołu pełnił
również rolę programisty (za wyjątkiem kierownika projektu). Po analizie wymagań
zawartych w~instrukcji do~projektu dostępnej pod~adresem:
\emph{http://www.ia.pw.edu.pl/~tkruk/edu/rsob2014/lab/rso\_projekt2014.txt}
ustalono następujący harmonogram prac:
\begin{itemize}
  \item \textbf{4~marca 2014}: ustalenie podziału ról oraz organizacji
  środowiska programistycznego projektu,
  \item \textbf{11~marca 2014}: ustalenie funkcjonalności oferowanych przez
  system, ustalenie zawartości pliku konfiguracyjnego, ustalenie zarysu architektury systemu,
  \item \textbf{14~marca 2014}: ustalenie scenariuszy wysyłania i pobierania
  pliku,
  \item \textbf{18~marca 2014}: ustalenie scenariuszy uruchomienia i zamnięcia
  systemu,
  \item \textbf{25~marca 2014}: ustalenie funkcjonalności realizowanych w fazie
  drugiej,
  \item \textbf{1~kwietnia 2014}: przygotowanie wstępnej wersji dokumentacji,
  \item \textbf{4~kwietnia 2014}: opracowanie interfejsów komunikacyjnych
  udostępnianych przez serwer nazewniczy (master), serwer przechowujący (slave) oraz klienta;
  opracowanie komunikatów wymienianych między masterem i slave’em, masterem i
  klientem oraz slave’em i klientem,
  \item \textbf{6~kwietnia 2014}: przygotowanie dokumentacji spełniającej
  wszystkie wymogi dotyczące I etapu projektu,
  \item \textbf{22~kwietnia 2014}: implementacja operacji wysyłania i pobierania
  danych z systemu przy założeniu bezawaryjności urządzeń oraz komunikacji między
  nimi; nie będzie zaimplementowana replikacja; ustalenie pełnego planu testów;
  opracowanie końcowej demonstracji projektu,
  \item \textbf{29~kwietnia 2014}: implementacja operacji usuwania danych,
  \item \textbf{4~maja 2014}: opracowanie wersji dokumentacji spełniającej
  wszystkie wymogi dotyczące II etapu projektu,
  \item \textbf{16~maja 2014}: zapewnienie odpowiedniej niezawodności
  przechowywania danych (m.in. implementacja replikacji plików, uwzględnienie możliwych do
  wystąpienia sytuacji awaryjnych),
  \item \textbf{27~maja 2014}: usunięcie wykrytych błędów istniejących w
  systemie,
  \item \textbf{2~czerwca 2014}: przygotowanie pełnej dokumentacji projektu
  spełniającej wymogi dotyczące III etapu projektu.
\end{itemize}
